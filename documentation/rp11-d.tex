% -*- mode: LaTeX; fill-column: 96 -*-
%
% RP11-D Programming Manual for the QSIC/USIC

\chapter{RP11-D}

The RP11-D is the implementation of the RP11 disk controller inside the QSIC
and USIC. The RP11 never existed on the QBUS at all, but it is being
supported there because it's a nice simple controller, and can be very simply
extended to provide very large disks.

The RP11-D has been extended to support 22-bit addressing, both on the QSIC
(for the 22-bit QBUS), and on the USIC (where it's notionally a MASSBUS
device), to be able to have access to the $2^{22}$ bytes of main memory
supported by the ENABLE+ functionality available on the USIC.

The extended address needs more bits alongside the existing Bus Address
Register; these are stored in a new register, the 'Extended Address Register'
(RPXA). To find a place for this, the RP11-D takes advantage of the fact that
the original RP11-C hardware actually responds from $776700_8$ up to
$776736_8$, but there are no actual registers from $776700_8$ to
$776706_8$, so we can put any additional registers needed in that range.

We've also extended it further to allow for much larger disks. This is done by the
simple expedient of extending the track and cylinder fields into all the adjacent
unused bits, and allowing the sector, head, and cylinder fields of the disk address
to take on any value that fits. (These bits are not enabled unless the RP11-D is
configured for extended packs.) The result is 28~bits of linear block address, or a
maximum disk size of $2^{37}$ bytes or 128~GiBytes.  Obviously you'll need the
ability to modify your disk driver to take advantage of this.  Another added register
allows the device driver to see the configured pack size of each disk pack.

\section{Configuration}

The RP11-D has a set of configuration registers that take the place of the jumpers
and DIP switches of earlier disk controllers.  Additionally, there are configuration
registers that make up a ``load table'' that maps logical disk drives to disk packs
located on the various storage media.

This block of registers is referenced from the top-level configuration block as
described in \S\ref{conf}, page \pageref{conf}.  In this way, multiple RP11-Ds can
each have its own set of configuration registers and they can move around in the
configuration address space without needing to redefine anything here.  Addresses
shown here are relative to the beginning of this block of RP11-D configuration.

\subsection{Device}

\regaddr{Start+0}

\begin{register16}
  \bit{15}{ENA}
  \bit{14}{Q22}
  \bit{13}{EXT}
  \bits{0}{12}{Base Address}
\end{register16}

\begin{bittable}
  00-12 & Base Address & The base address of the RP11-D's registers in the I/O page
  on the QBUS or Unibus.  The default for the first RP11 is $776700_8$.\\

  13 & Extended & Enables extended disk packs.  This breaks compatibility with
  previous RP11s but allows for much larger disks if you're able to modify your disk
  driver. \\

  14 & Q22 & Enables 22-bit operation, otherwise the RP11-D acts as an 18-bit device.
  On the QBUS, Q22 would be normal while on the Unibus, 18-bits would be normal.
  
  Selecting Q22 on the Unibus while the Enable+ is also enabled, directs the RP11-D
  to DMA directly to 22-bit memory with physical addresses, not going through the
  Enable+ Unibus mapping registers.  An ersatz MASSBUS disk, if you will.  Enabling
  22-bit addressing on the Unibus without the Enable+ is currently undefined. \\

  15 & Enable & Enables this RP11-D. \\
\end{bittable}

\regaddr{Start+1}

\begin{register16}
  \bits{11}{15}{---}
  \bits{9}{10}{INT PRI}
  \bits{0}{8}{Interrupt Vector}
\end{register16}

\begin{bittable}
  00-08 & Interrupt Vector & The interrupt vector to use.  The default for the first
  RP11 is $254_8$. \\

  09-10 & INT PRI & The interrupt priority.  The default is interrupt priority
  5. \newline {\tt
    \begin{tabular}{ll}
      Priority 4 & 00 \\
      Priority 5 & 01 \\
      Priority 6 & 10 \\
      Priority 7 & 11 \\
  \end{tabular}} \\
\end{bittable}

\subsection{Load Table}

The Load Table immediately follows the device configuration and tells the RP11-D
about simulated ``disk packs'' that are loaded into its ``drives''.  It has eight
entries, corresponding to the eight disk drives the disk controller supports.  Each
entry is four consecutive configuration words.

\regaddr{Start+2,6,10,14,18,22,26,30}

\begin{register16}
  \bit{15}{ENA}
  \bit{14}{EXT}
  \bit{13}{RP03}
  \bits{3}{12}{---}
  \bits{0}{2}{SD}
\end{register16}
\begin{register16}
  \bits{0}{15}{Offset Low}
\end{register16}
\begin{register16}
  \bits{0}{15}{Offset High}
\end{register16}
\begin{register16}
  \bits{0}{15}{Size}
\end{register16}

\begin{bittable}
  00-02 & SD & The Storage Device the pack lives on.  See the table at
  \S\ref{storagedevice}, page \pageref{storagedevice}. \\

  13 & RP03 & If this is not an extended disk pack, this specifies an RP03 disk (406
  cylinders, \verb|~|40MB) otherwise an RP02 disk (203 cylinders, \verb|~|20MB). \\

  14 & EXT & This is an extended disk pack.  Only allowed if the RP11-D is configured
  to allow extended packs.\footnote{Considering using the File Unsafe error (RPDS bit
    09) to indicate that an extended pack was configured on a non-extended RP11-D.} \\

  15 & ENA & This pack is enabled.  Whether the pack shows as loaded depends on both
  this bit and whether the associated storage device is active. \\

   & Offset & The offset, in blocks, of the pack from the beginning of its Storage
  Device. \\

   & Size & For extended disk packs, this is the largest cylinder number (one less than the
  number of cylinders).  It is set to 0 for legacy RP02 or RP03 disk packs, allowing for a
  mix of legacy and extended disk packs on the same RP11-D controller. \\
\end{bittable}

On extended disk packs, the RP11-D has 4~bits to specify the sector address and
8~bits for the track.  All values are used (16 sectors per track and 256~tracks per
cylinder) so there are $2^{12}$ or 4,096~sectors per cylinder.

The smallest disk supported is 4~MiB with 2 cylinders (since a size of 0, 1 cylinder,
specifies a legacy disk pack) while the largest is 65,536~cylinders or 128~GiB.


\section{Programming}

The RP11-D is substantially compatible with the RP11-C, with extensions for
extended addressing, the optional extended disk addresses, and again, some
different meanings to error bits to better match the flash media the
QSIC/USIC uses for storage devices.

Initiate functions (Idle, Seek and Home Seek) only require short periods;
execute functions (the rest) tie up the controller until it is finished
with the operation. No other operation (initiate or execute) can be
started until the execute function has completed. Plain Read and Write
include an implicit seek, if needed; the Seek command allows specifying
the desired head (track) as well.

After reading or writing the last sector in a track, the RP11-D automatically
advances to the next track; if the track that overflowed was the last track
in the cylinder, the cylinder automatically advances to the next cylinder. If
the cylinder that overflowed was the last cylinder, End of Pack in the RPER
is set.

Neither the 36-bit mode, nor header commands (i.e. Header bit in the RPCS
set), nor either parity, is supported.

The default interrupt vector is $254_8$ and the interrupt priority 
is 5. Both the interrupt priority and vector are configurable.

\section{Registers}

The address shown for each register is the default address for the
first RP11 controller.

\subsection{Pack Size Register (RPPS)}
\regaddr{776704}

\bigskip
If the larger pack size is not enabled, this register reads as 0, and writing
has no effect, like on the RP11-C.

If it is enabled, this register give the size of the pack currently mounted
on the drive selected by the 'Drive Select' field of the RPCS.  If a
regular RP02 or RP03 pack is loaded on a drive, the pack size will
show as 0 and the disk addressing for that disk will be the regular
cylinder/track/sector rather than a linear block address.

\begin{register16}
  \bits{0}{15}{Pack Size}
\end{register16}

\begin{bittable}
  00-15 & Pack Size (PS) & Contains the largest valid cylinder number of
  the drive selected by the 'Drive Select' field of the RPCS; read-only. \\
\end{bittable}

\subsection{Extended Address Register (RPXA)}
\regaddr{776706}

\bigskip
If addressing is set to 18-bits, this register reads as 0 and writing has no
effect, like on the RP11-C.

If addressing is set to 22-bits, this register extends the Bus
Address register to a full 22-bits.  On the UNIBUS, this only makes
sense in the presence of the ENABLE+ and the address is then a
physical address rather than being mapped by the ENABLE+. On the QBUS
it's always a physical address anyway.

\begin{register16}
  \bits{0}{5}{BAE}
  \bits{6}{15}{0}
\end{register16}

\begin{bittable}
  00-05 & Bus Address Extension (BAE) & If 22-bit addressing is
  enabled, these bits extend the Bus Address Register to 22-bits.
  Bits 00 and 01 are duplicates of MEX (bits 04 and 05 of RPCS) and
  may be read or written through either register. \\
\end{bittable}

\subsection{Drive Status Register (RPDS)}
\regaddr{776710}

\begin{register16}
  \bit{0}{ATTN 0}
  \bit{1}{ATTN 1}
  \bit{2}{ATTN 2}
  \bit{3}{ATTN 3}
  \bit{4}{ATTN 4}
  \bit{5}{ATTN 5}
  \bit{6}{ATTN 6}
  \bit{7}{ATTN 7}
  \bit{8}{SU WP}
  \bit{9}{SU FU}
  \bit{10}{SU SU}
  \bit{11}{SU SI}
  \bit{12}{HNF}
  \bit{13}{SU RP03}
  \bit{14}{SU OL}
  \bit{15}{SU RDY}
\end{register16}

The Attention bits are read-write (and may be written by a 'write byte' bus
cycle), the rest are read-only.

\begin{bittable}
  00-07 & Drive Attention & Set when the drive completes a seek.\\
  
  08 & Selected Unit Write Protected & Set when the selected drive is in
  write-protected mode. \\

  09 & Selected Unit File Unsafe & Unused, set to 0.\footnote{Could be used to
    indicate storage device initialization failure, perhaps.} \\

  10 & Selected Unit Seek Underway & Unused, set to 0.\footnote{May depend on
    what we do with seeks.} \\

  11 & Selected Unit Seek Incomplete & Set to 0; seeks always complete. \\

  12 & Header Not Found & Unused, set to 0. \\

  13 & Selected Unit RP03 & Set to 1 to indicate this is an RP03. \\

  14 & Selected Unit Online & Set to 1 when i) an SD card has been installed;
    ii) it has successfully completed initialization; iii) a pack partition
    on that card has been assigned to this drive. \\

  15 & Selected Unit Ready & Set to 1 as with SU OL, except during a read or
	write operation to this disk, when is it 0. \\
\end{bittable}

\subsection{Error Register (RPER)}
\regaddr{776712}

% Argh!! Why the inverse numbering from the previous?
\begin{register16}
  \bit{15}{WPV}
  \bit{14}{FUV}
  \bit{13}{NXC}
  \bit{12}{NXT}
  \bit{11}{NXS}
  \bit{10}{\tiny PROG}
  \bit{9}{\tiny FMTE}
  \bit{8}{\tiny MODE}
  \bit{7}{LPE}
  \bit{6}{WPE}
  \bit{5}{\tiny CSME}
  \bit{4}{\tiny TIMEE}
  \bit{3}{WCE}
  \bit{2}{\tiny NXME}
  \bit{1}{EOP}
  \bit{0}{\tiny DSK ERR }
\end{register16}

The RPER is a read-only register (except in maintenance mode, which
is not currently supported).

\begin{bittable}
  00 & Disk Error & OR of HNF and SU SI, so always 0. \\

  01 & End of Pack & Indicates that, during a Read, Write, Read Check,
    or Write Check function, operations on sector $11_8$, track $23_8$,
    and cylinder $625_8$ were finished, and the RPWC has not yet overflowed.
    This is essentially an attempt to overflow out of a drive. \\

  02 & Nonexistent Memory & Set if memory does not respond
  within the bus timeout on a memory cycle. \\

  03 & Write Check Error & Indicates that the data comparison
    didn't match during a Write Check function.\footnote{Not yet
    implemented.} \\

  04 & Timing Error & Unused, set to 0. \\

  05 & Checksum Error & Indicates a checksum error while reading
  data during a Read Check or Read function. The RP11-D does not do
  its own checksums on the data and this bit reflects the checksum
  from the SD Card or USB checksum.\footnote{Not yet implemented.} \\

  06 & Word Parity Error & Unused, set to 0. \\

  07 & Longitudinal Parity Error & Unused, set to 0. \\

  08 & Mode Error & Unused, set to 0. \\

  09 & Format Error & Unused, set to 0. \\

  10 & Programming Error & OR of transfer attempted with the RPWC
    set to 0; an operation was attempted on a drive which was not online;
    an operation was attempted while another was still in progress. \\

  11 & Non-existent Sector & Indicates that an attempt was made
  to initiate an operation to a sector larger than $11_8$. \\

  12 & Non-existent Track & Indicates that an attempt was made
  to initiate an operation to a track larger than $23_8$. \\

  13 & Non-existent Cylinder & Indicates that an attempt was made
  to initiate a transfer to a cylinder larger than $625_8$. \\

  14 & File Unsafe Violation & Unused, set to 0. \\

  15 & Write Protect Violation & Set if an attempt is made to
    write to a disk that is currently write-protected.\footnote{Not
    yet implemented.} \\
\end{bittable}

\subsection{Control Status Register (RPCS)}
\regaddr{776714}

\begin{register16}
  \bit{15}{ERR}
  \bit{14}{HE}
  \bit{13}{AIE}
  \bit{12}{\tiny MODE}
  \bit{11}{HDR}
  \bits{8}{10}{DRV SEL}
  \bit{7}{RDY}
  \bit{6}{IDE}
  \bits{4}{5}{MEX}
  \bits{1}{3}{COM}
  \bit{0}{GO}
\end{register16}

\begin{bittable}
  00 & Go & When set, causes the RP11-D to act on the function
    contained in bits 01 through 03 of the RPCS; when set, it
    sets Not Ready bit (which does not appear to be in any register;
    perhaps the Ready bit in the CSR is the inversion of that).
    Write-only, always reads as 0 (it is not stored; rather, the level during
    the bus write operation is used as a pulse). \\

  01-03 & Function & The function to be executed when Go is
  set.\newline
  {\tt
    \begin{tabular}{ll}
      Idle/Reset & 000 \\
      Write & 001 \\
      Read & 010 \\
      Write Check & 011 \\
      Seek & 100 \\
      Write (no seek) & 101 \\
      Home Seek & 110 \\
      Read (no seek) & 111 \\
  \end{tabular}} \\

  04-05 & Memory Extended Address & A 2-bit extension to RPBA giving
    an 18-bit bus address. If 22-bit addresses are enabled,
    these two bits are replicated as bits 00 and 01 of RPXA. \\

  06 & Interrupt on Done (Error) Enable & When set, causes an interrupt
    to be issued on various conditions.\footnote{Should audit the code
    and list all the conditions that can generate an interrupt.}\\

  07 & Ready & Controller is ready to perform a new function;
    read-only. \\

  08-10 & Drive Select & Specify the drive for any controller command. \\

  11 & Header & Not applicable to the QSIC/USIC.\footnote{Currently the
    Header bit is ignored but it probably should generate some sort of
    error.} \\

  12 & Mode & Not applicable to the QSIC/USIC.\footnote{Currently the
    Mode bit is ignored but it probably should generate some sort of
    error.} \\

  13 & Attention Interrupt Enable & Allows the RP11-D to generate an
    interrupt when any Attention bit (in RPDS) is set.\footnote{Not yet
    implemented.} \\

  14 & Hard Error & Set when any error other than a data error is set;
    read-only. \\

  15 & Error (ERR) & Set when any error is set; read-only. \\
\end{bittable}

\subsection{Word Count Register (RPWC)}
\regaddr{776716}

\begin{register16}
  \bits{0}{15}{Word Count}
\end{register16}

\begin{bittable}
  00-15 & Word Count & The 2's complement of the number of words to be
  transferred by a function.  The register increments by one after
  each word transfer.  When the register overflows to 0, the transfer
  is completed and the RP11 function is terminated. \\
\end{bittable}

\subsection{Bus Address Register (RPBA)}
\regaddr{776720}

\begin{register16}
  \bits{1}{15}{Bus Address}
  \bit{0}{0}
\end{register16}

\begin{bittable}
  00-15 & BA00-BA15 & The low 16-bits of the bus address to be used for data
  transfers.  The MEX bits (bits 04 and 05 of RPCS) extend the address to 18-bits
  and, if enabled, the BAE bits (bits 00-05 of RPXA) extend the address to
  22-bits. Bit 00 is always 0 as all transfers are a full word. \\
\end{bittable}

\subsection{Cylinder Address Register (RPCA)}
\regaddr{776722}

\begin{register16}
  \bits{0}{8}{Cylinder Address}
  \bits{9}{15}{Extended Cylinder Address}
\end{register16}

When in RP02/RP03 emulation mode, bits 0-8 are read-write, and 9-15 are
unused.

\begin{bittable}
  00-08 & Cylinder 00-08 & The cylinder number when emulating an
  RP02/03. \\

  09-15 & Extended Cylinder 09-15 & The high bits of the cylinder number,
  when emulating an extended pack. \\
\end{bittable}

\subsection{Disk Address Register (RPDA)}
\regaddr{776724}

\begin{register16}
  \bits{0}{3}{Sector Address}
  \bits{4}{7}{Current Sector}
  \bits{8}{12}{Track Address}
  \bits{13}{15}{Extended Track}
\end{register16}

Used for all operations other than Home Seek. Seek uses only the
Track Address.

\begin{bittable}
  00-03 & Sector Address & The disk sector to be addressed for
  the next function. \\

  04-07 & Current Sector & Notionally, the current sector address of the
	currently selected drive; read-only. On the RP11-D, connected to
	a free-running counter; the data is of no validity. \\

  08-12 & Track 00-04 & The track number, when emulating an RP02/03. \\

  13-15 & Extended Track 05-07 & The extended track number, when emulating
	an extended pack. \\
\end{bittable}

\subsection{Maintenance 1 Register (RPM1)}
\regaddr{776726}

\begin{register16}
  \bits{0}{15}{Unused}
\end{register16}

This register is currently unimplemented in the RP11-D.

\begin{bittable}
  00-15 & Unused & Unused \\
\end{bittable}

\subsection{Maintenance 2 Register (RPM2)}
\regaddr{776730}

This register is currently unimplemented in the RP11-D.

\begin{register16}
  \bits{0}{15}{Unused}
\end{register16}

\begin{bittable}
  00-15 & Unused & Unused \\
\end{bittable}

\subsection{Maintenance 3 Register (RPM3)}
\regaddr{776732}

This register is currently unimplemented in the RP11-D.

\begin{register16}
  \bits{0}{15}{Unused}
\end{register16}

\begin{bittable}
  00-15 & Unused & Unused \\
\end{bittable}

\subsection{Selected Unit Cylinder Address(SUCA)}
\regaddr{776734}

\begin{register16}
  \bits{0}{8}{Cylinder Address}
  \bits{9}{15}{Extended Cylinder Address}
\end{register16}

This register appears to be read-only. (Need to check the prints.)

\begin{bittable}
  00-08 & Cylinder 00-08 & Contains the cylinder address of the selected
    drive (I think!) when emulating an RP02/03. \\

  09-15 & Extended Cylinder 09-15 & Contains the extended cylinder address
    when emulating an extended pack. \\
\end{bittable}

\subsection{Silo Memory Buffer Register (SILO)}
\regaddr{776736}

\begin{register16}
  \bits{0}{15}{Silo end}
\end{register16}

\begin{bittable}
  00-15 & Silo & This register, when enabled by maintenance
  (currently un-implemented) allows reading from and writing to
  the FIFO connecting the RP11-D to its storage device. \\
\end{bittable}

