% -*- LaTeX -*-
%
% RK11-F Programming Manual for the QSIC/USIC

\chapter{RK11-F}
\section{Intro}

The RK11-F is the implementation of the RK11 disk controller inside
the QSIC and USIC.  It is substantially compatible with the RK11-D
with extensions for extended addressing on the QBUS and some different
meanings to error bits to better match the flash media the QSIC/USIC
uses for storage devices.

\section{Programming}

The address shown for each register is the default address for the
first RK11 controller.  The default interrupt vector is $220_8$ and
the interrupt priority (on the QBUS) is 4.  These are all
configurable.\footnote{Assuming we include two RK11s by default, I
  should add the default values for the second RK11 too.}

\subsection{Drive Status Register (RKDS)}
\regaddr{777400}

\begin{register16}
  \bits{0}{3}{\scriptsize Sector Counter}
  \bit{4}{\scriptsize SC=\\SA}
  \bit{5}{\scriptsize WPS}
  \bit{6}{\tiny R/W/S\\RDY}
  \bit{7}{\scriptsize DRY}
  \bit{8}{\scriptsize SOK}
  \bit{9}{\scriptsize SIN}
  \bit{10}{\scriptsize DRU}
  \bit{11}{\scriptsize RK05}
  \bit{12}{\scriptsize DPL}
  \bits{13}{15}{\scriptsize Drive Ident}
\end{register16}

\begin{bittable}
  00-03 & Sector Counter (SC) & The current sector address of the
  selected drive. On the QSIC/USIC, this is just a free-running
  counter clocked at 312.5kHz ($3.2\mu s$).\footnote{Check me that
    this is about the right speed.}  All disks share a single Sector
  Counter. \\
  
  04 & Sector Counter Equals Sector Address (SC=SA) & Set when the
  Sector Counter is equal to the Sector Address (RKDA 3-0). \\

  05 & Write Protect Status (WPS) & Set when the selected disk is in
  the write-protected mode. \\

  06 & Read/Write/Seek Ready\newline (R/W/S RDY) & Indicates a storage
  device is loaded and ready to accept commands.  In the QSIC/USIC, a
  storage device may be serving multiple disks at once and so unable
  to accept commands right now because it's otherwise busy.  However,
  the disk controller can still accept a command and it will wait
  until the storage device is ready.\footnote{This is a description of
    what it does right now.  As I wrote this, I realized that another
    possible implementation, perhaps better, would for it to be the
    AND of DRY (Drive Ready) and RDY (Control Ready).} \\

  07 & Drive Ready (DRY) & A storage device is loaded for the selected
  disk. \\

  08 & Sector Counter OK (SOK) & Unused, set to 1. \\

  09 & Seek Incomplete (SIN) & Set to 0.  Seeks always complete. \\

  10 & Drive Unsafe (DRU) & Unused, set to 0.\footnote{Could be used to
    indicate storage device initialization failure, perhaps.} \\

  11 & RK05 & Set to 1 to indicate this is an RK05. \\

  12 & Drive Power Low (DPL) & Unused, set to 0. \\

  13-15 & Identification of Drive (ID) & Set to the drive number that
  caused an interrupt.\footnote{Currently I set this from state
    CMD\_DONE regardless of whether I generate an interrupt or not.
    Also, I should check that it sets ID for all commands that could
    generate an interrupt.} \\
\end{bittable}

\subsection{Error Register (RKER)}
\regaddr{777402}

\begin{register16}
  \bit{15}{\scriptsize DRE}
  \bit{14}{\scriptsize OVR}
  \bit{13}{\scriptsize WLO}
  \bit{12}{\scriptsize SKE}
  \bit{11}{\scriptsize PGE}
  \bit{10}{\scriptsize NXM}
  \bit{9}{\scriptsize DLT}
  \bit{8}{\scriptsize TE}
  \bit{7}{\scriptsize NXD}
  \bit{6}{\scriptsize NXC}
  \bit{5}{\scriptsize NXS}
  \bits{2}{4}{\scriptsize Unused}
  \bit{1}{\scriptsize CSE}
  \bit{0}{\scriptsize WCE}
\end{register16}

\begin{bittable}
  00 & Write Check Error (WCE) & Indicates that the data comparison
  didn't match during a Write Check function.\footnote{Not yet
    implemented.} \\

  01 & Checksum Error (CSE) & Indicates a checksum error while reading
  data during a Read Check or Read function.  The RK05-F does not do
  its own checksums on the data and this bit reflects the checksum
  from the SD Card or USB checksum.\footnote{Not yet implemented.} \\

  02-04 & Unused  \\

  05 & Nonexistent Sector (NXS) & Indicates that an attempt was made
  to initiate a transfer to a sector larger than $13_8$. \\

  06 & Nonexistent Cylinder (NXC) & Indicates that an attempt was made
  to initiate a transfer to a cylinder larger than $312_8$. \\

  07 & Nonexistent Disk (NXD) & Indicates that an attempt was made to
  initiate a function on a nonexistent drive.\footnote{Not yet
    implemented.} \\

  08 & Timing Error (TE) & Unused, set to 0. \\

  09 & Data Late (DLT) & Unused, set to 0. \\

  10 & Nonexistent Memory (NXM) & Set if memory does not respond
  within the but timeout on the memory cycle. \\

  11 & Programming Error (PGE) & Unused, set to 0. \\

  12 & Seek Error (SKE) & Unused, set to 0. \\

  13 & Write Lockout Violation (WLO) & Set if an attempt is made to
  write to a disk that is currently write-protected.\footnote{Not
    yet implemented.} \\

  14 & Overrun (OVR) & Indicates that, during a Read, Write, Read
  Check, or Write Check function, operations on sector $13_8$, surface
  1, and cylinder address $312_8$ were finished, and the RKWC has not
  yet overflowed.  This is essentially an attempt to overflow out of a
  disk drive. \\

  15 & Drive Error (DRE) & Unused, set to 0. \\
\end{bittable}

\subsection{Control Status Register (RKCS)}
\regaddr{777404}

\begin{register16}
  \bit{15}{\scriptsize ERR}
  \bit{14}{\scriptsize HE}
  \bit{13}{\scriptsize SCP}
  \bit{12}{\scriptsize ---}
  \bit{11}{\scriptsize IBA}
  \bit{10}{\scriptsize FMT}
  \bit{9}{\scriptsize EXB}
  \bit{8}{\scriptsize SSE}
  \bit{7}{\scriptsize RDY}
  \bit{6}{\scriptsize IDE}
  \bits{4}{5}{\scriptsize MEX}
  \bits{1}{3}{\scriptsize FUNC}
  \bit{0}{\scriptsize GO}
\end{register16}

\begin{bittable}
  00 & GO & When set, causes the RK11-F to act on the function
  contained in bits 01 through 03 of the RKCS.\footnote{The RK11-D and
    RK11-E manual lists GO as write-only.  The RK11-F currently allows
    it to be read.  Oh wait, I know why they did that.  I need to fix
    the code.} \\

  01-03 & Function & The function to be executed when GO is
  set.\newline
  {\tt
    \begin{tabular}{ll}
      Control Reset & 000 \\
      Write & 001 \\
      Read & 010 \\
      Write Check & 011 \\
      Seek & 100 \\
      Read Check & 101 \\
      Drive Reset & 110 \\
      Write Lock & 111 \\
  \end{tabular}}  \\

  04-05 & Memory Extension (MEX) & A 2-bit extension to RKBA giving an
  18-bit bus address.  If 22-bit addresses are enabled (QSIC only),
  these two bits are replicated as bits 00 and 01 of RKXA. \\

  06 & Interrupt on Done Enable (IDE) & When set, causes an interrupt
  to be issued on various condition.\footnote{Should audit the code
    and list all the conditions that can generate an interrupt.}  The
  interrupt priority and vector are configurable. \\

  07 & Control Ready (RDY) & Control is ready to perform a function. \\

  08 & Stop on Soft Error & Currently not implemented. \\

  09 & Extra bit (EXB) & Unused. \\

  10 & Format (FMT) & Not applicable to the
  QSIC/USIC.\footnote{Currently the FMT bit is ignored but it probably
    should generate some sort of error.} \\

  11 & Inhibit Incrementing the RKBA (IBA) & Inhibits the RKBA from
  incrementing during a normal transfer.  This allows data transfers
  to occur to or from the same memory location throughout the entire
  transfer operation. \\

  12 & Unused \\

  13 & Search Complete (SCP) & Indicates that the previous interrupt
  was the result of some previous Seek or Drive Reset function.
  Cleared at the initiation of any new function.\footnote{Not yet
    implemented.} \\

  14 & Hard Error & Set when any of RKER 05-15 are set. \\

  15 & Error (ERR) & Set when any of RKER is set. \\
\end{bittable}

\subsection{Word Count Register (RKWC)}
\regaddr{777406}

\begin{register16}
  \bits{0}{15}{\scriptsize Word Count}
\end{register16}

\begin{bittable}
  00-15 & Word Count & The 2's complement of the number of words to be
  transferred by a function.  The register increments by one after
  each word transfer.  When the register overflows to 0, the transfer
  is completed and the RK11 function is terminated. \\
\end{bittable}

\subsection{Current Bus Address Register (RKBA)}
\regaddr{777410}

\begin{register16}
  \bits{1}{15}{\scriptsize Bus Address}
  \bit{0}{\scriptsize 0}
\end{register16}

\begin{bittable}
  00-15 & BA00-BA15 & The low 16-bits of the bus address to be used
  for data transfers.  The MEX bits (bits 04 and 05 of RKCS) extend
  the address to 18-bits and, if enabled, the BAE bits (bits 00-05 of
  RKXA) extend the address to 22-bits (QSIC only).  Bit 00 is always 0
  as all transfers are a full word. \\
\end{bittable}

\subsection{Disk Address Register (RKDA)}
\regaddr{777412}

\begin{register16}
  \bits{0}{3}{\scriptsize Sector Address}
  \bit{4}{\scriptsize SUR}
  \bits{5}{12}{\scriptsize Cylinder Address}
  \bits{13}{15}{\scriptsize Drive Select}
\end{register16}

\begin{bittable}
  00-03 & Sector Address (SA) & The disk sector to be addressed for
  the next function. \\

  04 & Surface (SUR) & Upper or lower surface has no meaning for SD
  cards or USB flash drives so this maps to just another bit of
  cylinder addressing. \\

  05-12 & Cylinder Address\newline (CYL ADDR) & The cylinder address
  currently being selection.  The largest valid cylinder is
  $312_8$. \\

  13-15 & Drive Select (DR SEL) & The logical drive number currently
  being selected. \\
\end{bittable}


\subsection{Extended Address Register (RKXA)}
\regaddr{777414}

\bigskip
On the RK11-C this register was a maintenance register and on the
RK11-D it was unused.  If addressing is set to 18-bits, this register
reads as 0 and writing has no effect, like on the RK11-D.

If addressing is set to 22-bits, this register extends the Bus
Address register to a full 22-bits.  On the Unibus, this only makes
sense in the presence of the ENABLE+ and the address is then a
physical address rather than being mapped by the ENABLE+.  On the QBUS
it's always a physical address anyway.

\begin{register16}
  \bits{0}{5}{\scriptsize BAE}
  \bits{6}{15}{\scriptsize 0}
\end{register16}

\begin{bittable}
  00-05 & Bus Address Extension (BAE) & If 22-bit addressing is
  enabled, these bits extend the Bus Address Register to 22-bits.
  Bits 00 and 01 are duplicates of MEX (bits 04 and 05 of RKCS) and
  may be read or written through either register. \\
\end{bittable}


\subsection{Data Buffer Register (RKDB)}
\regaddr{777416}

\begin{register16}
  \bits{0}{15}{\scriptsize Data Buffer}
\end{register16}

\begin{bittable}
  00-15 & Data Buffer (DB00-DB15) & This register reads from the read
  end of the FIFO connecting the RK11-F to its storage device.
  Writing to the Data Buffer has no effect. \\
\end{bittable}

