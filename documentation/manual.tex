% -*- LaTeX -*-
%
% QSIC/USIC Manual

\documentclass[12pt]{book}

\usepackage{array}
\usepackage{longtable}
%\usepackage{graphicx}
\usepackage[hidelinks=true]{hyperref}

% -*- LaTeX -*-
%
% For building register diagrams

\newenvironment{register36}
  { \noindent
    \setlength{\dimen0}{0.0272\textwidth}
    \setlength{\unitlength}{\dimen0}
    \begin{picture}(36, 3)
      % make the box outline
      \put(0, 2){\line(1, 0){36}}
      \put(0, 0){\line(1, 0){36}}
      \put(0,0){\line(0,1){2}}
      \put(36,0){\line(0,1){2}}
      % put the numbers on top
      \count255 = 0
      \loop
        \put(\the\count255, 2.3){\makebox[\dimen0][c]
	  {\tiny\number\count255}}
	\ifnum\count255 < 35
	\advance\count255 by 1
	\repeat

      \newcommand{\bit}[2] {
	\put(##1,0){\line(0,1){2}}
	\put(##1,0)
	    {\parbox[b][2\dimen0][c]{\dimen0}
	      {\begin{center} \tiny ##2 \end{center}}}
	    \count255 = ##1
	    \advance\count255 by 1
	    \put(\the\count255,0){\line(0,1){2}}  
      }
      	
    \newcommand{\bits}[3] {
      \put(##1,0){\line(0,1){2}}
      \count255 = ##2
      \advance\count255 by 1
      \put(\the\count255,0){\line(0,1){2}}
      \advance\count255 by -##1 % get the width of the field
      \put(##1,0)
	  {\parbox[b][2\dimen0][c]{\the\count255\dimen0}
	    {\centering \tiny ##3 }}
    }
  }
  {\end{picture} \smallskip}
	 
\newenvironment{register16}
  { \noindent
    \setlength{\dimen0}{0.0605\textwidth}
    \setlength{\unitlength}{\dimen0}
    \newcount\pos
    \begin{picture}(16, 2)
      % make the box outline
      \put(0, 1.1){\line(1,0){16}}
      \put(0, 0){\line(1,0){16}}
      \put(0, 0){\line(0,1){1.1}}
      \put(16,0){\line(0,1){1.1}}
      % put the numbers on top
      \count255 = 15
      \loop
        \pos = 15
        \advance\pos by -\count255
        \put(\pos, 1.2){\makebox[\dimen0][c]
	  {\tiny \ifnum \count255 < 10 0\fi
            \number\count255}}
        \put(\pos, 0.9){\line(0,1){.2}}
	\ifnum\count255 > 0
	\advance\count255 by -1
	\repeat

    \newcommand{\bit}[2] {
      \pos = 15
      \advance\pos by -##1
      \put(\pos,0){\line(0,1){1.1}}
      \put(\pos,0)
	  {\parbox[b][1.1\dimen0][c]{\dimen0}
	    {\begin{center} ##2 \end{center}}}
      \advance\pos by 1
      \put(\pos,0){\line(0,1){1.1}}  
    }
      	
    \newcommand{\bits}[3] {
      \pos = 15
      \advance\pos by -##2
      \put(\pos,0){\line(0,1){1.1}}
      \count255 = 15
      \advance\count255 by -##1
      \advance\count255 by 1
      \put(\the\count255,0){\line(0,1){1.1}}
      \advance\count255 by -\pos % get the width of the field
      \put(\pos,0)
	  {\parbox[b][1.1\dimen0][c]{\the\count255\dimen0}
	    {\centering \tiny ##3 }}
    }
  }
  {\end{picture} \smallskip}



\newcommand{\code}[1]{\textsf{#1}}
\newcommand{\regaddr}[1]{{\hspace{3em} \small\tt Address = #1}}

% Gives me a non-justified paragraph inside a table
\newcolumntype{P}[1]{>{\raggedright\arraybackslash}p{#1}}
% Gives me a \tt column inside a table
\newcolumntype{B}[0]{>{\tt\arraybackslash}l}

% The table used throughout for describing bits in registers.
\newenvironment{bittable}
               { \begin{longtable}{BP{0.22\textwidth}p{0.6\textwidth}}
                   {\bf Bit} & {\bf Designation} & {\bf Description
                     and Operation}  \endhead }
               { \end {longtable}}
                 
\includeonly{
  qsic,
  rk11-f,
  rp11-d
}

\title{QBUS/Unibus Storage Interface Card \\ (QSIC/USIC) \\ Programming Manual}
\author{David Bridgham \\
  Noel Chiappa}

\begin{document}
%\pagestyle{empty}
\maketitle
\pagenumbering{roman}
\tableofcontents\newpage

\pagenumbering{arabic}
\pagestyle{headings}

% -*- LaTeX -*-
%
% QSIC/USIC Programming and Configuration

\chapter{QBUS/UNIBUS Storage and Input/Output Card}
\section{Introduction}

The QSIC and USIC are cards which provide emulation of various QBUS and
UNIBUS controllers and disks, using MicroSD cards or (eventually) USB disks
as the actual storage media. They use NOS original bus drivers (DS8641s),
and then level converters to interface to a modern FPGA.

Eventually, they are likely be extended to allow emulation of other
controllers, e.g. the Interlan NI1010 and NI2010 Ethernet, via devices
plugged into the USB port.

They currently implement upwardly compatible, but extended, versions of the
old DEC RK11 and RP11 disk controllers. Although the emulation is not exact,
it is good enough that un-modified operating system images (UNIX V6) are able
to boot and run.

The emulation limitations are in part because many of the control register
bits only make sense with an actual physical drive; also, exact emulation,
including delays (e.g. for now non-existent seeks) would limit the
performance obtainable. (It is possible some systems that will not work
without better emulation of such delays; if so, an option could be added to
better emulate them.)

The QSIC is a dual-height QBUS card; it can hold two $\mu$SD cards,
allowing direct card-\textgreater card backup. The controllers are
denominated as the RKV11-F and RPV11-D; the extension is allowing DMA
to the entire $2^{22}$ byte QBUS address space.

The USIC is the same functionality in a quad-height SPC card for the UNIBUS.
The USIC will optionally adds the Able ENABLE functionality, which allows
processors with only 18-bit addressing to have access to $2^{22}$ bytes of
memory.

When this is enabled, the RK11-F and RP11-D (as they are denominated here)
can be set to be `MASSBUS' controllers (notionally), with full direct access
to the entire memory (which is on the USIC), without going through a UNIBUS
Map. With that turned off, they emulate the originals; i.e. they do DMA
cycles on the UNIBUS.

Both cards have provision for adding indicator panels, as close as possible
to the DEC originals, to display internal state and datal; this will help
invoke the feel of the older machines. They might even be useful for
debugging from time to time!

\section{Basic operation}

The space on the $\mu$SD cards (`storage devices') is divided into `packs',
described by a `pack table' on the card (because it applies only to that
card), which gives their location and size. Packs can be `loaded' on
`drives'; in other words, everything works much like the original hardware.

(The term `mount' is reserved for the operation of letting the operating
system add a pack to the visible file system - again, like the existing UNIX,
etc, terminology.)

There are also `load tables', which record which packs are loaded on which
drives; a non-volatile instance of these allows a system to be cold-booted
without going through a pack loading phase.

A storage device is `inserted' into a $\mu$SD slot; removing one without
previously un-mounting (and un-loading) the packs is an error which can
damage storage contents, just as switching a physical pack without
un-mounting it would have on the original hardware.

Removing a storage device will auto-unload any packs still loaded. Before any
further disk operations can happen, any packs on a new storage device which
are to be used have to be loaded; attempting to use them without that step
will produce `disk not loaded' faults (e.g. clears `Drive Ready' on the RK11).

Each controller supports the maximum 8 drives of the original. It will be
possible to configure more than one instance of each controller, should
simultaneous access to more packs/drives be needed.

\section{Configuration}
\label{conf}

The QSIC and USIC contain, potentially, a multitude of devices all of
which need configuration.  If we did it in the traditional manner with
jumpers and DIP switches, it would be a mess indeed.  Therefore,
configuration of these devices is handled through two I/O registers
which give access to a series of internal configuration registers
inside each of the emulated devices.

This configuration may be saved to internal flash memory where it will
be restored at startup.  Some configuration information is more
dynamic, namely the disk pack load tables, and needs to be re-computed
at each boot.\footnote{Still to be designed.}

\subsection{Bus Registers}

Access to the internal configuration registers is through two I/O
registers on the UNIBUS or QBUS located at $777720_8$ and $777722_8$.
The first register is the address register.  Setting this selects
which internal configuration register the second register accesses.
Reading or writing the second bus register then accesses the specified
configuration register.

\subsection{Top-Level Configuration Table}

The internal configuration begins at fixed location 0.  It gives some
information about the USIC or QSIC and then indexes all of the rest of
the configuration for the rest of the devices.


\begin{register16}
  \bits{14}{15}{Type}
  \bit{13}{FPGA Dev}
  \bit{12}{Soft Dev}
  \bit{11}{Save}
  \bits{4}{10}{---}
  \bits{0}{3}{Conf Vers}
\end{register16}
\regaddr{0}

\begin{bittable}
  14-15 & Type & Type of board.\newline
  {\tt
    \begin{tabular}{rl}
      USIC & 00 \\
      QSIC & 01 \\
  \end{tabular}} \\

  13 & FPGA Dev & The FPGA version shown is under development.  This
  bit will be cleared when it is a released version. \\

  12 & Soft Dev & The Software version shown is under development.  This
  bit will be cleared when it is a released version. \\

  11 & Save & Set this bit to cause the current configuration to be
  saved to flash memory.  While saving, this bit will read as 1.
  Do not modify the configuration while saving is in progress. \\

  0-3 & Conf Vers & Version of the configuration format.  This
  document describes version 0. \\
\end{bittable}

\begin{register16}
  \bits{8}{15}{FPGA Major Version}
  \bits{0}{7}{FPGA Minor Version}
\end{register16}
\regaddr{1}

\begin{register16}
  \bits{8}{15}{Software Major Version}
  \bits{0}{7}{Software Minor Version}
\end{register16}
\regaddr{2}

\begin{register16}
  \bits{8}{15}{Controller Count}
  \bits{0}{7}{Storage Device Count}
\end{register16}
\regaddr{3}

\begin{bittable}
  8-15 & Controller Count & A count of how many entries are in the
  controller table that follows. \\

  0-7 & Storage Device Count & A count of how many entries are in the
  storage devices table that follows the controller table.
\end{bittable}

Beginning at word 4, there is a table of Controllers followed by a
table of Storage Devices.  Each entry in the tables is one word long.

\begin{register16}
  \bits{11}{15}{Type}
  \bits{0}{10}{Index}
\end{register16}

\begin{bittable}
  11-15 & Type & The type of device referenced.  For controllers,
  types are:\newline
  {\tt
    \begin{tabular}{ll}
      Indicator Panels & 0 \\
      RK11-F & 1 \\
      RP11-D & 2 \\
      Enable+ & 3 \\
      Interlan 1010 & 4 \\
  \end{tabular}}\newline
  
  For Storage Devices:\newline
  {\tt
    \begin{tabular}{ll}
      SD Card & 0 \\
      RAM Disk & 1 \\
      USB & 2 \\
  \end{tabular}}\\

  0-10 & Index & The start address of the configuration block for the
  specified device.
\end{bittable}


\subsection{Storage Devices}

The storage devices have no configuration but they show up in the
configuration system as a way of reporting status and diagnostics.

\subsubsection{SD Card}

\begin{register16}
  \bit{15}{CD}
  \bit{14}{V2}
  \bit{13}{HC}
  \bit{12}{RDY}
  \bit{11}{RD}
  \bit{10}{WR}
  \bits{6}{9}{Error Code}
  \bits{0}{5}{Size}
\end{register16}

\begin{bittable}
  15 & CD & Card Detect \\
  14 & V2 & SD Version 2 \\
  13 & HC & High Capacity \\
  12 & RDY & The device is ready. \\
  11 & RD & The card is in the middle of a read operation. \\
  10 & WR & The card is in the middle of a write operation. \\
  6-9 & Error Code & If non-zero, the SD card controller is indicating an
  error.\footnote{Will fill out a table of error codes once they're
    set in the code.} \\
  0-5 & Size & The $log_2$ of the size of the SD Card in 512\ byte
  blocks.
\end{bittable}

\subsubsection{RAM Disk}

\begin{register16}
  \bits{6}{15}{---}
  \bits{0}{5}{Memory Size}
\end{register16}

\begin{bittable}
  0-5 & Size & The $log_2$ of the size of memory in 512\ byte blocks.
  Zero (0) indicates there is no RAM available.
\end{bittable}

\subsubsection{USB Device}

This is mostly a placeholder for now, since we have not even begun to
implement USB devices.

\begin{register16}
  \bit{15}{ACT}
  \bits{10}{14}{---}
  \bits{6}{9}{Error Code}
  \bits{0}{5}{Size}
\end{register16}

\begin{bittable}
  15 & ACT & This USB device is active. \\
  6-9 & Error Code & If non-zero, the USB Controller is indicating an
  error. \\
  0-5 & Size & The $log_2$ of the size of the USB device in 512\ byte blocks.
\end{bittable}

\subsubsection{Load Tables}
\label{storagedevice}

The load tables in each disk controller need to specify on which
storage device a disk pack resides.  These are the values that are
used.  Note that the SD Cards are physical locations while the
different USB devices are logical.

{\tt
  \begin{tabular}{rl}
    SD Card 0 & 0 \\
    SD Card 1 & 1 \\
    RAM Disk & 2 \\
    USB 0 & 3 \\
    USB 1 & 4 \\
    USB 2 & 5 \\
    USB 3 & 6 \\
    USB 4 & 7 \\
\end{tabular}} \\


\section{RAM disks}

In addition to storing pack images on removable media, the QSIC/USIC also
supports `RAM drives' - packs which are kept in on-board RAM on the card. The
contents do not survive power cycles, but RAM drives are desirable for uses
which do a lot of writing, none of which needs to survive a system re-boot,
e.g. swapping and paging.

The reason is that $\mu$SD cards (unlike the original magnetic media) cannot
perform un-limited write cycles; they wear out after a finite number of
writes. So, if there is no need to keep writes over re-boots (e.g. swapping
or paging data), those operations should be assigned to a RAM disk for
long-term use. On most operating systems, it is fairly easy to configure them
to do swapping/paging to particular devices.

\subsection{uSD card selection}

There have been reports of cheap commodity $\mu$SD cards failing after relatively
small numbers of writes. So, these should be avoided; also, backups of the
data on $\mu$SD cards should be kept fairly meticulously. There are ``industrial
grade'' $\mu$SD cards available, which are more robust than good consumer-grade
cards, so those are perhaps worth using.

\subsection{UNIX pipes}

The UNIX pipe device can also generate lots of ``temporary'' writes.
Unfortunately, on the early versions of UNIX, pipes are created on the root
device - probably the last place you want them, if there are $\mu$SD card issues!

This is simple to fix, though; in pipe(), in pipe.c, change the line:

	ip = ialloc(rootdev);

to

	ip = ialloc(pipedev);

and then go into c.c and add a line underneath the declaration of ``rootdev''
to add a ``pipedev''.

Don't forget that you will need to create a file system on any RAM pack you
are using to hold pipes, before you can use a pipe, though! Probably the
safest thing is to start the system with pipedev set to the root device, and
then reset pipedev once the pipe pack has a filesystem on it. (This is quite
OK, because once a pipe is created, it stays on the device it was created on,
and it's possible to have open pipes on more than one device.) A program to
set `pipedev' is available for V6.

\section{Upgrading drivers}

The devices on QSIC, and USIC with Enable+ on, can do DMA to all 22 bits of
address space, but are program compatible for 18 bits (i.e. existing software
will run, but can only use the low $2^{18}$ bytes of memory). To use more
than $2^{18}$ bytes - generally only needed to use the QSIC/USIC devices for
swapping/paging - the device driver will have to be fixed - not too
complicated, but only if one has the capability!

The 18-bit program compatability is, however, useful for extending operating
systems to use the 22-bit capability. (For systems like UNIX, which restrict
block device I/O to buffers in the low $2^{16}$ bytes of memory, the
22-capability is not needed if they are not being used for swapping/paging.)
Just simply boot the OS with only $2^{18}$ bytes of memory; modify the device
driver to be able to use the 22-bit addressing capability; and restart. (This
technique was used to give Unix V6 access to the 22-bit capability.)

NOTE: On OSs that auto-size memory, if you boot the system with more than
256KB of memory, if it tries to swap to high memory with the existing 18-bit
driver, either i) the system will crash if it knows that RK11's and/or RP11's
are 18-bit only (well, technically, RKV11-D's are actually only 16 bits), or
ii) the transfer will go someplace else in memory from where the OS thought
it was going to go. (E.g. UNIX V6 would do this - although the address in the
I/O request is 22 bits long, the existing RK11/RP11 drivers only look at the
low 18.) The solution is to boot the system with only 256KB of memory
installed; then update the driver to be able to do 22 bit DMA; then the rest
of the memory can be added back.

\chapter{Indicator Panels}

The QSIC and USIC support indicator panels which look exactly like the old
DEC ones. These comprise (like the originals) a bezel, a captioned inlay, a
light shield, and a PCB holding the lamps (``warm white'' LEDs, which look
identical to the original units, when seen through the inlay); the whole
mounts to a $19''$ rack.

As on the originals, the bezel and inlay go on the standard ``latch moldings''
(the flat plastic units with the two posts with a spherical ball on the top)
mounted to the rack; the light shield (``Benelex'' in DEC parts jargon) mounts
to the rack with a pair of brackets, and the lamp PCB mounts to that.

Although the new panels are mechanically compatible with the originals (so
that original bezels, inlays, etc can be used, and vice versa), the hardware
interface to the lamp PCB is totally different (bit-serial via a 4-wire
interface - data, clock, latch and ground), rather than `wire per bulb', as
in the originals. As a result, the PCBs are also completely different. The
new PCBs are limited in size (for ease of fabrication); each only holds 12
columns of lamps, and a set of 3 plug together to make the full 36-wide array
of the originals.

The maximum number of panels a single QSIC/USIC can support is not yet
determined, but should be at least 4.

We can supply complete units (mostly newly-fabricated), but we have only a
limited supply of original bezels.

\section{Configuration}

The configuration registers allow the user to set how many indicator
panels are attached and what should be displayed on them.

All the indicator panels are driven by a single, differential serial
line running at approximately 100kHz.  This serial line is
daisy-chained from one panel to the next.  The more indicator panels
you configure, the slower their update.  The current configuration
allows for up to seven but that maximum has not yet been tested.

\begin{register16}
  \bits{12}{15}{Count}
  \bits{8}{11}{Panel 1}
  \bits{4}{7}{Panel 2}
  \bits{0}{3}{Panel 3}
\end{register16}
\regaddr{Start+0}

\begin{register16}
  \bits{12}{15}{Panel 4}
  \bits{8}{11}{Panel 5}
  \bits{4}{7}{Panel 6}
  \bits{0}{3}{Panel 7}
\end{register16}
\regaddr{Start+1}

\begin{bittable}
  & Count & How many panels are active.\\

  & Panel \# & The type of each panel.  That is, what to display.
  Panel 1 is first in the chain. \newline
  {\tt \begin{tabular}{rll}
      lamptest & 0 & All lights on \\
      Bus Monitor & 1 & Unibus or QBUS\footnote{Also includes some SD Card and USB
      status lights.} \\
      RK11 \#0 & 2 \\
      RK11 \#1 & 3 \\
      RP11 \#0 & 4 \\
      RP11 \#1 & 5 \\
      Enable+ & 6 \\
      Interlan & 7\footnote{Do we even want a full indicator panel for
        the Interlan Ethernet board?  Maybe just grab two or three
        lights off the bus monitor display.} \\
      debugging & 15 \\
  \end{tabular}} \\
\end{bittable}



\section{Inlays}

Although we could do exact duplicates of original panels, the old DEC panels
have lots of lights that don't make sense without an actual physical disk
(e.g. `write current on'), and also leave off other ones that would be useful
(e.g. memory address). So, we default to a new layout we have designed, which
makes the best use of the available set of lamps.

The new layout works with both the RK11 and RP11; the RK11 does not drive all
the disk address lights, but is otherwise identical.

We might be able to provide exact copies of the old ones for people who are
crazy for authenticity; it will require custom FPGA loads to drive their
lamps correctly, though.

In addition to the RK/RP inlays, there is also an inlay to monitor QBUS
activity.


% -*- LaTeX -*-
%
% RK11-F Programming Manual for the QSIC/USIC

\chapter{RK11-F}
\section{Intro}

The RK11-F is the implementation of the RK11 disk controller inside
the QSIC and USIC.  It is substantially compatible with the RK11-D
with extensions for extended addressing on the QBUS and some different
meanings to error bits to better match the flash media the QSIC/USIC
uses for storage devices.

\section{Programming}

The address shown for each register is the default address for the
first RK11 controller.  The default interrupt vector is $220_8$ and
the interrupt priority (on the QBUS) is 4.  These are all
configurable.\footnote{Assuming we include two RK11s by default, I
  should add the default values for the second RK11 too.}

\subsection{Drive Status Register (RKDS)}
\regaddr{777400}

\begin{register16}
  \bits{0}{3}{\scriptsize Sector Counter}
  \bit{4}{\scriptsize SC=\\SA}
  \bit{5}{\scriptsize WPS}
  \bit{6}{\tiny R/W/S\\RDY}
  \bit{7}{\scriptsize DRY}
  \bit{8}{\scriptsize SOK}
  \bit{9}{\scriptsize SIN}
  \bit{10}{\scriptsize DRU}
  \bit{11}{\scriptsize RK05}
  \bit{12}{\scriptsize DPL}
  \bits{13}{15}{\scriptsize Drive Ident}
\end{register16}

\begin{bittable}
  00-03 & Sector Counter (SC) & The current sector address of the
  selected drive. On the QSIC/USIC, this is just a free-running
  counter clocked at 312.5kHz ($3.2\mu s$).\footnote{Check me that
    this is about the right speed.}  All disks share a single Sector
  Counter. \\
  
  04 & Sector Counter Equals Sector Address (SC=SA) & Set when the
  Sector Counter is equal to the Sector Address (RKDA 3-0). \\

  05 & Write Protect Status (WPS) & Set when the selected disk is in
  the write-protected mode. \\

  06 & Read/Write/Seek Ready\newline (R/W/S RDY) & Indicates a storage
  device is loaded and ready to accept commands.  In the QSIC/USIC, a
  storage device may be serving multiple disks at once and so unable
  to accept commands right now because it's otherwise busy.  However,
  the disk controller can still accept a command and it will wait
  until the storage device is ready.\footnote{This is a description of
    what it does right now.  As I wrote this, I realized that another
    possible implementation, perhaps better, would for it to be the
    AND of DRY (Drive Ready) and RDY (Control Ready).} \\

  07 & Drive Ready (DRY) & A storage device is loaded for the selected
  disk. \\

  08 & Sector Counter OK (SOK) & Unused, set to 1. \\

  09 & Seek Incomplete (SIN) & Set to 0.  Seeks always complete. \\

  10 & Drive Unsafe (DRU) & Unused, set to 0.\footnote{Could be used to
    indicate storage device initialization failure, perhaps.} \\

  11 & RK05 & Set to 1 to indicate this is an RK05. \\

  12 & Drive Power Low (DPL) & Unused, set to 0. \\

  13-15 & Identification of Drive (ID) & Set to the drive number that
  caused an interrupt.\footnote{Currently I set this from state
    CMD\_DONE regardless of whether I generate an interrupt or not.
    Also, I should check that it sets ID for all commands that could
    generate an interrupt.} \\
\end{bittable}

\subsection{Error Register (RKER)}
\regaddr{777402}

\begin{register16}
  \bit{15}{\scriptsize DRE}
  \bit{14}{\scriptsize OVR}
  \bit{13}{\scriptsize WLO}
  \bit{12}{\scriptsize SKE}
  \bit{11}{\scriptsize PGE}
  \bit{10}{\scriptsize NXM}
  \bit{9}{\scriptsize DLT}
  \bit{8}{\scriptsize TE}
  \bit{7}{\scriptsize NXD}
  \bit{6}{\scriptsize NXC}
  \bit{5}{\scriptsize NXS}
  \bits{2}{4}{\scriptsize Unused}
  \bit{1}{\scriptsize CSE}
  \bit{0}{\scriptsize WCE}
\end{register16}

\begin{bittable}
  00 & Write Check Error (WCE) & Indicates that the data comparison
  didn't match during a Write Check function.\footnote{Not yet
    implemented.} \\

  01 & Checksum Error (CSE) & Indicates a checksum error while reading
  data during a Read Check or Read function.  The RK05-F does not do
  its own checksums on the data and this bit reflects the checksum
  from the SD Card or USB checksum.\footnote{Not yet implemented.} \\

  02-04 & Unused  \\

  05 & Nonexistent Sector (NXS) & Indicates that an attempt was made
  to initiate a transfer to a sector larger than $13_8$. \\

  06 & Nonexistent Cylinder (NXC) & Indicates that an attempt was made
  to initiate a transfer to a cylinder larger than $312_8$. \\

  07 & Nonexistent Disk (NXD) & Indicates that an attempt was made to
  initiate a function on a nonexistent drive.\footnote{Not yet
    implemented.} \\

  08 & Timing Error (TE) & Unused, set to 0. \\

  09 & Data Late (DLT) & Unused, set to 0. \\

  10 & Nonexistent Memory (NXM) & Set if memory does not respond
  within the but timeout on the memory cycle. \\

  11 & Programming Error (PGE) & Unused, set to 0. \\

  12 & Seek Error (SKE) & Unused, set to 0. \\

  13 & Write Lockout Violation (WLO) & Set if an attempt is made to
  write to a disk that is currently write-protected.\footnote{Not
    yet implemented.} \\

  14 & Overrun (OVR) & Indicates that, during a Read, Write, Read
  Check, or Write Check function, operations on sector $13_8$, surface
  1, and cylinder address $312_8$ were finished, and the RKWC has not
  yet overflowed.  This is essentially an attempt to overflow out of a
  disk drive. \\

  15 & Drive Error (DRE) & Unused, set to 0. \\
\end{bittable}

\subsection{Control Status Register (RKCS)}
\regaddr{777404}

\begin{register16}
  \bit{15}{\scriptsize ERR}
  \bit{14}{\scriptsize HE}
  \bit{13}{\scriptsize SCP}
  \bit{12}{\scriptsize ---}
  \bit{11}{\scriptsize IBA}
  \bit{10}{\scriptsize FMT}
  \bit{9}{\scriptsize EXB}
  \bit{8}{\scriptsize SSE}
  \bit{7}{\scriptsize RDY}
  \bit{6}{\scriptsize IDE}
  \bits{4}{5}{\scriptsize MEX}
  \bits{1}{3}{\scriptsize FUNC}
  \bit{0}{\scriptsize GO}
\end{register16}

\begin{bittable}
  00 & GO & When set, causes the RK11-F to act on the function
  contained in bits 01 through 03 of the RKCS.\footnote{The RK11-D and
    RK11-E manual lists GO as write-only.  The RK11-F currently allows
    it to be read.  Oh wait, I know why they did that.  I need to fix
    the code.} \\

  01-03 & Function & The function to be executed when GO is
  set.\newline
  \begin{tabular}{cccl}
    {\bf Bit 3} & {\bf Bit 2} & {\bf Bit 1} & {\bf Operation} \\
     0 & 0 & 0 & Control Reset \\
     0 & 0 & 1 & Write \\
     0 & 1 & 0 & Read \\
     0 & 1 & 1 & Write Check \\
     1 & 0 & 0 & Seek \\
     1 & 0 & 1 & Read Check \\
     1 & 1 & 0 & Drive Reset \\
     1 & 1 & 1 & Write Lock \\
  \end{tabular} \\

  04-05 & Memory Extension (MEX) & A 2-bit extension to RKBA giving an
  18-bit bus address.  If 22-bit addresses are enabled (QSIC only),
  these two bits are replicated as bits 00 and 01 of RKXA. \\

  06 & Interrupt on Done Enable (IDE) & When set, causes an interrupt
  to be issued on various condition.\footnote{Should audit the code
    and list all the conditions that can generate an interrupt.}  The
  interrupt priority and vector are configurable. \\

  07 & Control Ready (RDY) & Control is ready to perform a function. \\

  08 & Stop on Soft Error & Currently not implemented. \\

  09 & Extra bit (EXB) & Unused. \\

  10 & Format (FMT) & Not applicable to the
  QSIC/USIC.\footnote{Currently the FMT bit is ignored but it probably
    should generate some sort of error.} \\

  11 & Inhibit Incrementing the RKBA (IBA) & Inhibits the RKBA from
  incrementing during a normal transfer.  This allows data transfers
  to occur to or from the same memory location throughout the entire
  transfer operation. \\

  12 & Unused \\

  13 & Search Complete (SCP) & Indicates that the previous interrupt
  was the result of some previous Seek or Drive Reset function.
  Cleared at the initiation of any new function.\footnote{Not yet
    implemented.} \\

  14 & Hard Error & Set when any of RKER 05-15 are set. \\

  15 & Error (ERR) & Set when any of RKER is set. \\
\end{bittable}

\subsection{Word Count Register (RKWC)}
\regaddr{777406}

\begin{register16}
  \bits{0}{15}{\scriptsize Word Count}
\end{register16}

\begin{bittable}
  00-15 & Word Count & The 2's complement of the number of words to be
  transferred by a function.  The register increments by one after
  each word transfer.  When the register overflows to 0, the transfer
  is completed and the RK11 function is terminated. \\
\end{bittable}

\subsection{Current Bus Address Register (RKBA)}
\regaddr{777410}

\begin{register16}
  \bits{1}{15}{\scriptsize Bus Address}
  \bit{0}{\scriptsize 0}
\end{register16}

\begin{bittable}
  00-15 & BA00-BA15 & The low 16-bits of the bus address to be used
  for data transfers.  The MEX bits (bits 04 and 05 of RKCS) extend
  the address to 18-bits and, if enabled, the BAE bits (bits 00-05 of
  RKXA) extend the address to 22-bits (QSIC only).  Bit 00 is always 0
  as all transfers are a full word. \\
\end{bittable}

\subsection{Disk Address Register (RKDA)}
\regaddr{777412}

\begin{register16}
  \bits{0}{3}{\scriptsize Sector Address}
  \bit{4}{\scriptsize SUR}
  \bits{5}{12}{\scriptsize Cylinder Address}
  \bits{13}{15}{\scriptsize Drive Select}
\end{register16}

\begin{bittable}
  00-03 & Sector Address (SA) & The disk sector to be addressed for
  the next function. \\

  04 & Surface (SUR) & Upper or lower surface has no meaning for SD
  cards or USB flash drives so this maps to just another bit of
  cylinder addressing. \\

  05-12 & Cylinder Address\newline (CYL ADDR) & The cylinder address
  currently being selection.  The largest valid cylinder is
  $312_8$. \\

  13-15 & Drive Select (DR SEL) & The logical drive number currently
  being selected. \\
\end{bittable}


\subsection{Extended Address Register (RKXA)}
\regaddr{777414}

\bigskip
This register is an incompatible extension to the RK11, available on
the QSIC only\footnote{Potentially we could have 22-bit addressing
  with the USIC if using the Able ENABLE functionality.  Definitely an
  incompatible extension.}, in order to give full 22-bit addressing.
On the RK11-C it was a maintenance register and on the RK11-D and
RK11-E it was unused.  If addressing is set to 18-bits, this register
reads as 0 and writing has no effect.

\begin{register16}
  \bits{0}{5}{\scriptsize BAE}
  \bits{6}{15}{\scriptsize ---}
\end{register16}

\begin{bittable}
  00-05 & Bus Address Extension (BAE) & If 22-bit addressing is
  enabled, these bits extend the Bus Address Register to 22-bits.
  Bits 00 and 01 are duplicates of MEX (bits 04 and 05 of RKCS) and
  may be read or written through either register. \\
\end{bittable}


\subsection{Data Buffer Register (RKDB)}
\regaddr{777416}

\begin{register16}
  \bits{0}{15}{\scriptsize Data Buffer}
\end{register16}

\begin{bittable}
  00-15 & Data Buffer (DB00-DB15) & This register reads from the read
  end of the FIFO connecting the RK11-F to its storage device.
  Writing to the Data Buffer has no effect. \\
\end{bittable}


% -*- LaTeX -*-
%
% RP11-D Programming Manual for the QSIC/USIC

\chapter{RP11-D}

The RP11-D is the implementation of the RP11 disk controller inside the QSIC
and USIC. The RP11 never existed on the QBUS at all, but it is being
supported there because it's a nice simple controller, and can be very simply
extended to provide very large disks.

The RP11-D has been extended to support 22-bit addressing, both on the QSIC
(for the 22-bit QBUS), and on the USIC (where it's notionally a MASSBUS
device), to be able to have access to the $2^{22}$ bytes of main memory
supported by the ENABLE+ functionality available on the USIC.

The extended address needs more bits alongside the existing Bus Address
Register; these are stored in a new register, the 'Extended Address Register'
(RPXA). To find a place for this, the RP11-D takes advantage of the fact that
the original RP11-C hardware actually responds from $776700_8$ up to
$776736_8$, but there are no actual registers from $776700_8$ to
$776706_8$, so we can put any additional registers needed in that range.

We've also extended it further to allow for much larger disks. This is done by the
simple expedient of extending the track and cylinder fields into all the adjacent
unused bits, and allowing the sector, head, and cylinder fields of the disk address
to take on any value that fits. (These bits are not enabled unless the RP11-D is
configured for extended packs.) The result is 28~bits of linear block address, or a
maximum disk size of $2^{37}$ bytes or 128~GiBytes.  Obviously you'll need the
ability to modify your disk driver to take advantage of this.  Another added register
allows the device driver to see the configured pack size of each disk pack.

\section{Configuration}

The RP11-D has a set of configuration registers that take the place of the jumpers
and DIP switches of earlier disk controllers.  Additionally, there are configuration
registers that make up a ``load table'' that maps logical disk drives to disk packs
located on the various storage media.

This block of registers is referenced from the top-level configuration block as
described in \S\ref{conf}, page \pageref{conf}.  In this way, multiple RP11-Ds can
each have its own set of configuration registers and they can move around in the
configuration address space without needing to redefine anything here.  Addresses
shown here are relative to the beginning of this block of RP11-D configuration.

\subsection{Device}

\regaddr{Start+0}

\begin{register16}
  \bit{15}{ENA}
  \bit{14}{Q22}
  \bit{13}{EXT}
  \bits{0}{12}{Base Address}
\end{register16}

\begin{bittable}
  15 & Enable & Enables this RP11-D. \\

  14 & Q22 & Enables 22-bit operation, otherwise the RP11-D acts as an 18-bit device.
  On the QBUS, Q22 would be normal while on the Unibus, 18-bits would be normal.
  
  Selecting Q22 on the Unibus while the Enable+ is also enabled, directs the RP11-D
  to DMA directly to 22-bit memory with physical addresses, not going through the
  Enable+ Unibus mapping registers.  An ersatz MASSBUS disk, if you will.  Enabling
  22-bit addressing on the Unibus without the Enable+ is currently undefined. \\

  13 & Extended & Enables extended disk packs.  This breaks compatibility with
  previous RP11s but allows for much larger disks if you're able to modify your disk
  driver. \\

  00-12 & Base Address & The base address of the RP11-D's registers in the I/O page
  on the QBUS or Unibus.  The default for the first RP11 is $776700_8$.\\

\end{bittable}

\regaddr{Start+1}

\begin{register16}
  \bits{11}{15}{---}
  \bits{9}{10}{INT PRI}
  \bits{0}{8}{Interrupt Vector}
\end{register16}

\begin{bittable}
  09-10 & INT PRI & The interrupt priority.  The default is interrupt priority
  5. \newline {\tt
    \begin{tabular}{ll}
      Priority 4 & 00 \\
      Priority 5 & 01 \\
      Priority 6 & 10 \\
      Priority 7 & 11 \\
  \end{tabular}} \\

  00-08 & Interrupt Vector & The interrupt vector to use.  The default for the first
  RP11 is $254_8$. \\
\end{bittable}

\subsection{Load Table}

The Load Table immediately follows the device configuration and tells the RP11-D
about simulated ``disk packs'' that are loaded into its ``drives''.  It has eight
entries, corresponding to the eight disk drives the disk controller supports.  Each
entry is four consecutive configuration words.

\regaddr{Start+2,6,10,14,18,22,26,30}

\begin{register16}
  \bit{15}{ENA}
  \bit{14}{EXT}
  \bit{13}{RP03}
  \bits{3}{12}{---}
  \bits{0}{2}{SD}
\end{register16}
\begin{register16}
  \bits{0}{15}{Offset Low}
\end{register16}
\begin{register16}
  \bits{0}{15}{Offset High}
\end{register16}
\begin{register16}
  \bits{0}{15}{Size}
\end{register16}

\begin{bittable}
  15 & ENA & This pack is enabled.  Whether the pack shows as loaded depends on both
  this bit and whether the associated storage device is active. \\

  14 & EXT & This is an extended disk pack.  Only allowed if the RP11-D is configured
  to allow extended packs.\footnote{Considering using the File Unsafe error (RPDS bit
    09) to indicate that an extended pack was configured on a non-extended RP11-D.} \\

  13 & RP03 & If this is not an extended disk pack, this specifies an RP03 disk (406
  cylinders, \verb|~|40MB) otherwise an RP02 disk (203 cylinders, \verb|~|20MB). \\

  00-02 & SD & The Storage Device the pack lives on.  See the table at
  \S\ref{storagedevice}, page \pageref{storagedevice}. \\

   & Offset & The offset, in blocks, of the pack from the beginning of its Storage
  Device. \\

   & Size & For extended disk packs, this is the largest cylinder number (one less than the
  number of cylinders).  It is set to 0 for legacy RP02 or RP03 disk packs, allowing for a
  mix of legacy and extended disk packs on the same RP11-D controller. \\
\end{bittable}

On extended disk packs, the RP11-D has 4~bits to specify the sector address and
8~bits for the track.  All values are used (16 sectors per track and 256~tracks per
cylinder) so there are $2^{12}$ or 4,096~sectors per cylinder.

The smallest disk supported is 4~MiB with 2 cylinders (since a size of 0, 1 cylinder,
specifies a legacy disk pack) while the largest is 65,536~cylinders or 128~GiB.


\section{Programming}

The RP11-D is substantially compatible with the RP11-C, with extensions for
extended addressing, the optional extended disk addresses, and again, some
different meanings to error bits to better match the flash media the
QSIC/USIC uses for storage devices.

Initiate functions (Idle, Seek and Home Seek) only require short periods;
execute functions (the rest) tie up the controller until it is finished
with the operation. No other operation (initiate or execute) can be
started until the execute function has completed. Plain Read and Write
include an implicit seek, if needed; the Seek command allows specifying
the desired head (track) as well.

After reading or writing the last sector in a track, the RP11-D automatically
advances to the next track; if the track that overflowed was the last track
in the cylinder, the cylinder automatically advances to the next cylinder. If
the cylinder that overflowed was the last cylinder, End of Pack in the RPER
is set.

Neither the 36-bit mode, nor header commands (i.e. Header bit in the RPCS
set), nor either parity, is supported.

The default interrupt vector is $254_8$ and the interrupt priority 
is 5. Both the interrupt priority and vector are configurable.

\section{Registers}

The address shown for each register is the default address for the
first RP11 controller.

\subsection{Pack Size Register (RPPS)}
\regaddr{776704}

\bigskip
If the larger pack size is not enabled, this register reads as 0, and writing
has no effect, like on the RP11-C.

If it is enabled, this register give the size of the pack currently mounted
on the drive selected by the 'Drive Select' field of the RPCS.  If a
regular RP02 or RP03 pack is loaded on a drive, the pack size will
show as 0 and the disk addressing for that disk will be the regular
cylinder/track/sector rather than a linear block address.

\begin{register16}
  \bits{0}{15}{Pack Size}
\end{register16}

\begin{bittable}
  00-15 & Pack Size (PS) & Contains the largest valid cylinder number of
  the drive selected by the 'Drive Select' field of the RPCS; read-only. \\
\end{bittable}

\subsection{Extended Address Register (RPXA)}
\regaddr{776706}

\bigskip
If addressing is set to 18-bits, this register reads as 0 and writing has no
effect, like on the RP11-C.

If addressing is set to 22-bits, this register extends the Bus
Address register to a full 22-bits.  On the UNIBUS, this only makes
sense in the presence of the ENABLE+ and the address is then a
physical address rather than being mapped by the ENABLE+. On the QBUS
it's always a physical address anyway.

\begin{register16}
  \bits{0}{5}{BAE}
  \bits{6}{15}{0}
\end{register16}

\begin{bittable}
  00-05 & Bus Address Extension (BAE) & If 22-bit addressing is
  enabled, these bits extend the Bus Address Register to 22-bits.
  Bits 00 and 01 are duplicates of MEX (bits 04 and 05 of RPCS) and
  may be read or written through either register. \\
\end{bittable}

\subsection{Drive Status Register (RPDS)}
\regaddr{776710}

\begin{register16}
  \bit{0}{ATTN 0}
  \bit{1}{ATTN 1}
  \bit{2}{ATTN 2}
  \bit{3}{ATTN 3}
  \bit{4}{ATTN 4}
  \bit{5}{ATTN 5}
  \bit{6}{ATTN 6}
  \bit{7}{ATTN 7}
  \bit{8}{SU WP}
  \bit{9}{SU FU}
  \bit{10}{SU SU}
  \bit{11}{SU SI}
  \bit{12}{HNF}
  \bit{13}{SU RP03}
  \bit{14}{SU OL}
  \bit{15}{SU RDY}
\end{register16}

The Attention bits are read-write (and may be written by a 'write byte' bus
cycle), the rest are read-only.

\begin{bittable}
  00-07 & Drive Attention & Set when the drive completes a seek.\\
  
  08 & Selected Unit Write Protected & Set when the selected drive is in
  write-protected mode. \\

  09 & Selected Unit File Unsafe & Unused, set to 0.\footnote{Could be used to
    indicate storage device initialization failure, perhaps.} \\

  10 & Selected Unit Seek Underway & Unused, set to 0.\footnote{May depend on
    what we do with seeks.} \\

  11 & Selected Unit Seek Incomplete & Set to 0; seeks always complete. \\

  12 & Header Not Found & Unused, set to 0. \\

  13 & Selected Unit RP03 & Set to 1 to indicate this is an RP03. \\

  14 & Selected Unit Online & Set to 1 when i) an SD card has been installed;
    ii) it has successfully completed initialization; iii) a pack partition
    on that card has been assigned to this drive. \\

  15 & Selected Unit Ready & Set to 1 as with SU OL, except during a read or
	write operation to this disk, when is it 0. \\
\end{bittable}

\subsection{Error Register (RPER)}
\regaddr{776712}

% Argh!! Why the inverse numbering from the previous?
\begin{register16}
  \bit{15}{WPV}
  \bit{14}{FUV}
  \bit{13}{NXC}
  \bit{12}{NXT}
  \bit{11}{NXS}
  \bit{10}{\tiny PROG}
  \bit{9}{\tiny FMTE}
  \bit{8}{\tiny MODE}
  \bit{7}{LPE}
  \bit{6}{WPE}
  \bit{5}{\tiny CSME}
  \bit{4}{\tiny TIMEE}
  \bit{3}{WCE}
  \bit{2}{\tiny NXME}
  \bit{1}{EOP}
  \bit{0}{\tiny DSK ERR }
\end{register16}

The RPER is a read-only register (except in maintenance mode, which
is not currently supported).

\begin{bittable}
  00 & Disk Error & OR of HNF and SU SI, so always 0. \\

  01 & End of Pack & Indicates that, during a Read, Write, Read Check,
    or Write Check function, operations on sector $11_8$, track $23_8$,
    and cylinder $625_8$ were finished, and the RPWC has not yet overflowed.
    This is essentially an attempt to overflow out of a drive. \\

  02 & Nonexistent Memory & Set if memory does not respond
  within the bus timeout on a memory cycle. \\

  03 & Write Check Error & Indicates that the data comparison
    didn't match during a Write Check function.\footnote{Not yet
    implemented.} \\

  04 & Timing Error & Unused, set to 0. \\

  05 & Checksum Error & Indicates a checksum error while reading
  data during a Read Check or Read function. The RP11-D does not do
  its own checksums on the data and this bit reflects the checksum
  from the SD Card or USB checksum.\footnote{Not yet implemented.} \\

  06 & Word Parity Error & Unused, set to 0. \\

  07 & Longitudinal Parity Error & Unused, set to 0. \\

  08 & Mode Error & Unused, set to 0. \\

  09 & Format Error & Unused, set to 0. \\

  10 & Programming Error & OR of transfer attempted with the RPWC
    set to 0; an operation was attempted on a drive which was not online;
    an operation was attempted while another was still in progress. \\

  11 & Non-existent Sector & Indicates that an attempt was made
  to initiate an operation to a sector larger than $11_8$. \\

  12 & Non-existent Track & Indicates that an attempt was made
  to initiate an operation to a track larger than $23_8$. \\

  13 & Non-existent Cylinder & Indicates that an attempt was made
  to initiate a transfer to a cylinder larger than $625_8$. \\

  14 & File Unsafe Violation & Unused, set to 0. \\

  15 & Write Protect Violation & Set if an attempt is made to
    write to a disk that is currently write-protected.\footnote{Not
    yet implemented.} \\
\end{bittable}

\subsection{Control Status Register (RPCS)}
\regaddr{776714}

\begin{register16}
  \bit{15}{ERR}
  \bit{14}{HE}
  \bit{13}{AIE}
  \bit{12}{\tiny MODE}
  \bit{11}{HDR}
  \bits{8}{10}{DRV SEL}
  \bit{7}{RDY}
  \bit{6}{IDE}
  \bits{4}{5}{MEX}
  \bits{1}{3}{COM}
  \bit{0}{GO}
\end{register16}

\begin{bittable}
  00 & Go & When set, causes the RP11-D to act on the function
    contained in bits 01 through 03 of the RPCS; when set, it
    sets Not Ready bit (which does not appear to be in any register;
    perhaps the Ready bit in the CSR is the inversion of that).
    Write-only, always reads as 0 (it is not stored; rather, the level during
    the bus write operation is used as a pulse). \\

  01-03 & Function & The function to be executed when Go is
  set.\newline
  {\tt
    \begin{tabular}{ll}
      Idle/Reset & 000 \\
      Write & 001 \\
      Read & 010 \\
      Write Check & 011 \\
      Seek & 100 \\
      Write (no seek) & 101 \\
      Home Seek & 110 \\
      Read (no seek) & 111 \\
  \end{tabular}} \\

  04-05 & Memory Extended Address & A 2-bit extension to RPBA giving
    an 18-bit bus address. If 22-bit addresses are enabled,
    these two bits are replicated as bits 00 and 01 of RPXA. \\

  06 & Interrupt on Done (Error) Enable & When set, causes an interrupt
    to be issued on various conditions.\footnote{Should audit the code
    and list all the conditions that can generate an interrupt.}\\

  07 & Ready & Controller is ready to perform a new function;
    read-only. \\

  08-10 & Drive Select & Specify the drive for any controller command. \\

  11 & Header & Not applicable to the QSIC/USIC.\footnote{Currently the
    Header bit is ignored but it probably should generate some sort of
    error.} \\

  12 & Mode & Not applicable to the QSIC/USIC.\footnote{Currently the
    Mode bit is ignored but it probably should generate some sort of
    error.} \\

  13 & Attention Interrupt Enable & Allows the RP11-D to generate an
    interrupt when any Attention bit (in RPDS) is set.\footnote{Not yet
    implemented.} \\

  14 & Hard Error & Set when any error other than a data error is set;
    read-only. \\

  15 & Error (ERR) & Set when any error is set; read-only. \\
\end{bittable}

\subsection{Word Count Register (RPWC)}
\regaddr{776716}

\begin{register16}
  \bits{0}{15}{Word Count}
\end{register16}

\begin{bittable}
  00-15 & Word Count & The 2's complement of the number of words to be
  transferred by a function.  The register increments by one after
  each word transfer.  When the register overflows to 0, the transfer
  is completed and the RP11 function is terminated. \\
\end{bittable}

\subsection{Bus Address Register (RPBA)}
\regaddr{776720}

\begin{register16}
  \bits{1}{15}{Bus Address}
  \bit{0}{0}
\end{register16}

\begin{bittable}
  00-15 & BA00-BA15 & The low 16-bits of the bus address to be used for data
  transfers.  The MEX bits (bits 04 and 05 of RPCS) extend the address to 18-bits
  and, if enabled, the BAE bits (bits 00-05 of RPXA) extend the address to
  22-bits. Bit 00 is always 0 as all transfers are a full word. \\
\end{bittable}

\subsection{Cylinder Address Register (RPCA)}
\regaddr{776722}

\begin{register16}
  \bits{0}{8}{Cylinder Address}
  \bits{9}{15}{Extended Cylinder Address}
\end{register16}

When in RP02/RP03 emulation mode, bits 0-8 are read-write, and 9-15 are
unused.

\begin{bittable}
  00-08 & Cylinder 00-08 & The cylinder number when emulating an
  RP02/03. \\

  09-15 & Extended Cylinder 09-15 & The high bits of the cylinder number,
  when emulating an extended pack. \\
\end{bittable}

\subsection{Disk Address Register (RPDA)}
\regaddr{776724}

\begin{register16}
  \bits{0}{3}{Sector Address}
  \bits{4}{7}{Current Sector}
  \bits{8}{12}{Track Address}
  \bits{13}{15}{Extended Track}
\end{register16}

Used for all operations other than Home Seek. Seek uses only the
Track Address.

\begin{bittable}
  00-03 & Sector Address & The disk sector to be addressed for
  the next function. \\

  04-07 & Current Sector & Notionally, the current sector address of the
	currently selected drive; read-only. On the RP11-D, connected to
	a free-running counter; the data is of no validity. \\

  08-12 & Track 00-04 & The track number, when emulating an RP02/03. \\

  13-15 & Extended Track 05-07 & The extended track number, when emulating
	an extended pack. \\
\end{bittable}

\subsection{Maintenance 1 Register (RPM1)}
\regaddr{776726}

\begin{register16}
  \bits{0}{15}{Unused}
\end{register16}

This register is currently unimplemented in the RP11-D.

\begin{bittable}
  00-15 & Unused & Unused \\
\end{bittable}

\subsection{Maintenance 2 Register (RPM2)}
\regaddr{776730}

This register is currently unimplemented in the RP11-D.

\begin{register16}
  \bits{0}{15}{Unused}
\end{register16}

\begin{bittable}
  00-15 & Unused & Unused \\
\end{bittable}

\subsection{Maintenance 3 Register (RPM3)}
\regaddr{776732}

This register is currently unimplemented in the RP11-D.

\begin{register16}
  \bits{0}{15}{Unused}
\end{register16}

\begin{bittable}
  00-15 & Unused & Unused \\
\end{bittable}

\subsection{Selected Unit Cylinder Address(SUCA)}
\regaddr{776734}

\begin{register16}
  \bits{0}{8}{Cylinder Address}
  \bits{9}{15}{Extended Cylinder Address}
\end{register16}

This register appears to be read-only. (Need to check the prints.)

\begin{bittable}
  00-08 & Cylinder 00-08 & Contains the cylinder address of the selected
    drive (I think!) when emulating an RP02/03. \\

  09-15 & Extended Cylinder 09-15 & Contains the extended cylinder address
    when emulating an extended pack. \\
\end{bittable}

\subsection{Silo Memory Buffer Register (SILO)}
\regaddr{776736}

\begin{register16}
  \bits{0}{15}{Silo end}
\end{register16}

\begin{bittable}
  00-15 & Silo & This register, when enabled by maintenance
  (currently un-implemented) allows reading from and writing to
  the FIFO connecting the RP11-D to its storage device. \\
\end{bittable}



\end{document}
