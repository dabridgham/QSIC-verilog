% -*- LaTeX -*-
%
% QSIC/USIC Programming and Configuration
% I checked, it was "Storage-I/O"

\chapter{QBUS/UNIBUS Storage and Input/Output Card}
\section{Introduction}

The QSIC and USIC are cards which provide emulation of various QBUS and
UNIBUS controllers and disks, using MicroSD cards or (eventually) USB disks
as the actual storage media. They use NOS original bus drivers (DS8641's),
and then level converters to interface to a modern FPGA.

Eventually, they are likely be extended to allow emulation of other
controllers, e.g. the Interlan NI1010 and NI2010 Ethernet, via devices
plugged into the USB port.

They currently implement upwardly compatible, but extended, versions of the
old DEC RK11 and RP11 disk controllers. Although the emulation is not exact,
it is good enough that un-modified operating system images (UNIX V6) are able
to boot and run.

The emulation limitations are in part because many of the control register
bits only make sense with an actual physical drive; also, exact emulation,
including delays (e.g. for now non-existent seeks) would limit the
performance obtainable. (It is possible some systems that will not work
without better emulation of such delays; if so, an option could be added to
better emulate them.)

The QSIC is a dual-height QBUS card; it can hold two uSD cards, allowing
direct card->card backup. The controllers are denominated as the RKV11-F and
RPV11-D; the extension is allowing DMA to the entire $2^{22}$ byte QBUS
address space.

The USIC is the same functionality, in a quad-height SPC card for the UNIBUS.
The USIC will optionally adds the Able ENABLE functionality, which allows
processors with only 18-bit addressing to have access to $2^{22}$ bytes of
memory.

When this is enabled, the RK11-F and RP11-D (as they are denominated here)
can be set to be 'MASSBUS' controllers (notionally), with full direct access
to the entire memory (which is on the USIC), without going through a UNIBUS
Map. With that turned off, they emulate the originals; i.e. they do DMA
cycles on the UNIBUS.

Both cards have provision for adding indicator panels, as close as possible
to the DEC originals, to display internal state and datal; this will help
invoke the feel of the older machines. They might even be useful for
debugging from time to time!

\section{Basic operation}

The space on the uSD cards ('storage devices') is divided into 'packs',
described by a 'pack table' on the card (because it applies only to that
card), which gives their location and size. Packs can be 'loaded' on
'drives'; in other words, everything works much like the original hardware.

(The term 'mount' is reserved for the operation of letting the operating
system add a pack to the visible file system - again, like the existing UNIX,
etc, terminology.)

There are also 'load tables', which record which packs are loaded on which
drives; a non-volatile instance of these allows a system to be cold-booted
without going through a pack loading phase.

A storage device is 'inserted' into a uSD slot; removing one without
previously un-mounting (and un-loading) the packs is an error which can
damage storage contents, just as switching a physical pack without
un-mounting it would have on the original hardware.

Removing a storage device will auto-unload any packs still loaded. Before any
further disk operations can happen, any packs on a new storage device which
are to be used have to be loaded; attempting to use them without that step
will produce 'disk not loaded' faults (e.g. clears 'Drive Ready' on the RK11).

Each controller supports the maximum 8 drives of the original. It will be
possible to configure more than one instance of each controller, should
simultaneous access to more packs/drives be needed.

\section{Configuration}

\begin{center}
  $\ll$ To Be Written. $\gg$
\end{center}

(The means for setting up the pack table, loading packs onto drives, etc are
not yet defined.) 

\section{RAM disks}

In addition to storing pack images on removable media, the QSIC/USIC also
supports 'RAM drives' - packs which are kept in on-board RAM on the card. The
contents do not survive power cycles, but RAM drives are desirable for uses
which do a lot of writing, none of which needs to survive a system re-boot,
e.g. swapping and paging.

The reason is that uSD cards (unlike the original magnetic media) cannot
perform un-limited write cycles; they wear out after a finite number of
writes. So, if there is no need to keep writes over re-boots (e.g. swapping
or paging data), those operations should be assigned to a RAM disk for
long-term use. On most operating systems, it is fairly easy to configure them
to do swapping/paging to particular devices.

\subsection{uSD card selection}

There have been reports of cheap commodity uSD cards failing after relatively
small numbers of writes. So, these should be avoided; also, backups of the
data on uSD cards should be kept fairly meticulously. There are "industrial
grade" uSD cards available, which are more robust than good consumer-grade
cards, so those are perhaps worth using.

\subsection{UNIX pipes}

The UNIX pipe device can also generate lots of "temporary" writes.
Unfortunately, on the early versions of UNIX, pipes are created on the root
device - probably the last place you want them, if there are uSD card issues!

This is simple to fix, though; in pipe(), in pipe.c, change the line:

	ip = ialloc(rootdev);

to

	ip = ialloc(pipedev);

and then go into c.c and add a line underneath the declaration of "rootdev"
to add a "pipedev".

Don't forget that you will need to create a file system on any RAM pack you
are using to hold pipes, before you can use a pipe, though! Probably the
safest thing is to start the system with pipedev set to the root device, and
then reset pipedev once the pipe pack has a filesystem on it. (This is quite
OK, because once a pipe is created, it stays on the device it was created on,
and it's possible to have open pipes on more than one device.) A program to
set 'pipedev' is available for V6.

\section{Upgrading drivers}

The devices on QSIC, and USIC with Enable+ on, can do DMA to all 22 bits of
address space, but are program compatible for 18 bits (i.e. existing software
will run, but can only use the low $2^{18}$ bytes of memory). To use more
than $2^{18}$bytes - generally only needed to use the QSIC/USIC devices for
swapping/paging - the device driver will have to be fixed - not too
complicated, but only if one has the capability!

The 18-bit program compatability is, however, useful for extending operating
systems to use the 22-bit capability. (For systems like UNIX, which restrict
block device I/O to buffers in the low $2^{16}$ bytes of memory, the
22-capability is not needed if they are not being used for swapping/paging.)
Just simply boot the OS with only $2^{18}$ bytes of memory; modify the device
driver to be able to use the 22-bit addressing capability; and restart. (This
technique was used to give Unix V6 access to the 22-bit capability.)

NOTE: On OSs that auto-size memory, if you boot the system with more than
256KB of memory, if it tries to swap to high memory with the existing 18-bit
driver, either i) the system will crash if it knows that RK11's and/or RP11's
are 18-bit only (well, technically, RKV11-D's are actually only 16 bits), or
ii) the transfer will go someplace else in memory from where the OS thought
it was going to go. (E.g. UNIX V6 would do this - although the address in the
I/O request is 22 bits long, the existing RK11/RP11 drivers only look at the
low 18.) The solution is to boot the system with only 256KB of memory
installed; then update the driver to be able to do 22 bit DMA; then the rest
of the memory can be added back.

\chapter{Indicator Panels}

The QSIC and USIC support indicator panels which look exactly like the old
DEC ones. These comprise (like the originals) a bezel, a captioned inlay, a
light shield, and a PCB holding the lamps ("warm white" LEDs, which look
identical to the original units, when seen through the inlay); the whole
mounts to a 19" rack.

As on the originals, the bezel and inlay go on the standard "latch moldings"
(the flat plastic units with the two posts with a spherical ball on the top)
mounted to the rack; the light shield ("Benelex" in DEC parts jargon) mounts
to the rack with a pair of brackets, and the lamp PCB mounts to that.

Although the new panels are mechanically compatible with the originals (so
that original bezels, inlays, etc can be used, and vice versa), the hardware
interface to the lamp PCB is totally different (bit-serial via a 4-wire
interface - data, clock, latch and ground), rather than 'wire per bulb', as
in the originals. As a result, the PCBs are also completely different. The
new PCBs are limited in size (for ease of fabrication); each only holds 12
columns of lamps, and a set of 3 plug together to make the full 36-wide array
of the originals.

The maximum number of panels a single QSIC/USIC can support is not yet
determined, but should be at least 4.

We can supply complete units (mostly newly-fabricated), but we have only a
limited supply of original bezels.

\section{Inlays}

Although we could do exact duplicates of original panels, the old DEC panels
have lots of lights that don't make sense without an actual physical disk
(e.g. 'write current on'), and also leave off other ones that would be useful
(e.g. memory address). So, we default to a new layout we have designed, which
makes the best use of the available set of lamps.

The new layout works with both the RK11 and RP11; the RK11 does not drive all
the disk address lights, but is otherwise identical.

We might be able to provide exact copies of the old ones for people who are
crazy for authenticity; it will require custom FPGA loads to drive their
lamps correctly, though.

In addition to the RK/RP inlays, there is also an inlay to monitor QBUS
activity.

